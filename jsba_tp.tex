%!TEX root = jsba_main.tex

%Titelseite
\newpage
\thispagestyle{empty}
\newcommand{\Rule}{\rule{\textwidth}{1mm}}

\begin{center}

\Rule\vspace{5mm}
\sffamily\bfseries\Huge
Similarity-based query answering for medical information systems
\vspace{1mm}\Rule
\vfill
\sffamily\bfseries\LARGE Bachelor Thesis
\vfill
\sffamily\bfseries\Large Jero Mario Schäfer\par Mat.nr.: 21552103\par Applied Informatics\par
\vfill

\raisebox{7mm}{Georg-August-University}
%\includegraphics[height=xxmm]{unilogo.pdf}
\raisebox{7mm}{Göttingen}\par
\vfill
\today
\end{center}


\newpage



\paragraph{Roter Faden für die Einleitung}

\begin{itemize}
    \item Distributed Database
    \begin{itemize}
        \item Was ist das?
        \item Vorteile, "Heute"
        \item ? Herausforderungen (-> in Bezug auf Anfragebeantwortung?)
        \item ? Apache Ignite?
    \end{itemize}
    
    \item Clustering-based Fragmentation
    \begin{itemize}
        \item Was ist das?
        \item ? Bezug zu MeSH? Im Beispiel mit MeSH?
        \item ? Ähnlichkeiten?
        \item ? medizinisches/klinisches Informationssystem?
        \item ? derived fragmentation
    \end{itemize}
    
    \item Flexible Query Answering (for SQL)
    \begin{itemize}
        \item Was ist das?
        \item Nutzen
        \item ? Beispiel? MeSH bzw. Informationssystem?
        \item ? Bezug zu DDBMS und Clustering-based fragmentation?\newline
                (\textrightarrow\,Inwiefern unterstützen o.g. Punkte das?)
    \end{itemize}
\end{itemize}



\paragraph{Theorieteil}

Alles etwas formaler? (Def. 1.: ..., Def. 2.: ... , Text ...)
Noch ein Schritt vor? Relationale Datenbanken, Relationales Datenmodell, ...

MeSH, UMLS ?

\begin{itemize}
    \item DDBs
    \begin{itemize}
        \item Begriffserklärungen/-definitionen\textrightarrow Glossar??? Abkürzungsverzeichnis???
        \item Architektur (Client-Server, Data/Computing Nodes, homogen/heterogen)
        \item Konzepte, Replikation und Fragmentierung (Datenverteilung)
        \item Distributed SQL, Query processing/optimization???
    \end{itemize}
    
    \item ...
\end{itemize}




% TODO-List
\listoftodos