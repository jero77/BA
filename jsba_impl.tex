%!TEX root = jsba_main.tex
% Implementation

\section{Implementation}
\label{sec:impl}

The technical details of the implementation are discussed in the following sections. First of all, there will be given an overview of the technical 
infrastructure in terms of a cluster of Ignite server nodes spread across physical machines in a network. Secondly, the clustering procedure adapting to the 
approximation algorithm for clustering as described in \citet{Gonzales1985} is explained and its functionality under different similarity thresholds and term
sets is illustrated. After this, the reference implementation of the system which is used as a benchmark to compare the developed approaches to is explained.
This is followed by a subsection investigating the two developed approaches technically. The last subsection deals with the challenges and benefits of query
processing in the different implementation alternatives and how similarity-based (Section~\ref{sec:meth_sbqa}) and flexible answering 
(Section~\ref{sec:meth_fqa}) can be realized in each of the implementations.


\subsection{Technical Infrastructure}

The distributed \citetalias{Ignite} databases underlying the sample medical information system that is evaluated here is represented by a cluster of three
Ignite nodes (cf. Section~\ref{sec:meth_ign}) where each of the nodes runs in a JVM and is hosted by one of three servers. The three servers are provided by the
"Gesellschaft für wissenschaftliche Datenverarbeitung mbH Göttingen" (GWDG) via a cloud server service and form the technical infrastructure to support data
storage and physical data distribution. Each server has a publicly available ip address and a private ip address that is only visible for the other servers 
in the same subnet. A total of 24~GB memory for the cloud is split equally among all machines and each server has 4 processors. The nodes can communicate via 
the links of the subnet they all belong to and each node can be accessed by its public ip address. Any Ignite client node can connect remotely to the cluster
and access the data stored by the server nodes. The server nodes have to be started manually with a certain configuration file that specifies all required
information, e.g. information about the distributed tables and the affinity functions to be used for partitioning (cf. Section~\ref{sec:meth_ign_panre}) and
collocation (cf. Section~\ref{sec:meth_ign_affc}).

\subsection{Clustering}
\label{sec:impl_clust}

The clustering, which is required as input for the used fragmentation strategy, is an adaptation of the approximation algorithm 
\citep{Gonzales1985} for the clustering that allows for computing several subsets of the MeSH disease terms (see Appendix~\ref{app:terms}) based on pairwise 
similarities instead of distances between the different terms such that all terms of the same subset are similar to one another. The similarity measure 
\citep{McInnes2009} that is used here and in \citet{Wiese2014} accepts pairs of terms of the MeSH taxonomy among others as input and can calculate the
pairwise similarities, $sim(a,b)$, with one of the implemented standard similarity measures. Those standard measures either base on the path length between
two nodes that represent the MeSH terms in the taxonomy or base on the information content \citep{Resnik1995}. Some of them are (among others):
\begin{itemize}
    \item the path length measure, calculated as $\frac{1}{length(path(a,b))}$, where $length(path(a,b))$ denotes the number of all nodes on the shortest path 
        between $a$ and $b$
    \item the Leacock \& Chodorow measure \citep{Leacock1998} is calculated as $-log(\frac{length(path(a,b))}{2*D})$, that again uses the path length and
        additionally the maximal depth D of all concepts in the taxonomy $T$, i.e. the maximal length of the shortest path from the root concept to any other
        concept, $\max\limits_{a \in T}~length(path(a, root(T))$, where $root(T)$ denotes the root of the taxonomy $T$
    \item the Wu \& Palmer measure \citep{Wu1994}, $\frac{2*depth(lcs(a,b))}{depth(a)+depth(b)}$, which is defined for the depth of the two nodes $a$ and 
        $b$ and the depth of their least common subsumer, $lcs(a,b)$, which is the most specific concept in the taxonomy that is an ancestor of $a$ and $b$
    \item the Resnik measure \citep{Resnik1995}, $IC(lcs(a,b))$ that takes for similarity calculation only the information content of the least common 
        subsumer of $ a$ and $b$ into account
\end{itemize}


For the sake of simplicity and due to the facts that, first of all, the UMLS::Similarity web interface
(\url{http://maraca.d.umn.edu/cgi-bin/umls_similarity/umls_similarity.cgi}, see also \citet{UMLS::Sim}) is rather slow (ca. 20s per pairwise similarity 
computation) and, secondly, an alternative local installation of the similarity module \citep{McInnes2009} would require a chargeable license for the SNOMED CT
clinical terminology (Systematized Nomenclature of Human and Veterinary Medicine – Clinical Terms, see \url{http://www.snomed.org/}) as it is part of the
underlying UMLS Metathesaurus, the implementation is here based on the path length measure and is restricted to a randomly chosen subset of maximal 100 terms 
of all disease describing MeSH terms. Additionally, the web interface can not be used dynamically due to the high computation time. Instead, all pairwise
similarities were precomputed for 10, 30 and 100 terms (Appendix~\ref{app:terms}) and stored in files that will be loaded and stored in-memory during run time
for fast access of the required similarity values. This is achieved by storing all pairwise similarities in a hash table and looking up the similarity values
at run time whenever needed.

The clustering itself is implemented as a Java method (see Appendix~\ref{app:java_clustering}) that takes as input a list of all possible values for the
relaxation attribute, i.e. a list of MeSH terms as strings, and returns a list of clusters, implemented as class "Cluster" (see
Appendix~\ref{app:java_cluster}). The code is an implementation of the pseudocode from Listing~2 in \cite{Wiese2014} and the similarity threshold alpha is
defined as class variable. The clustering starts with a single cluster (line~6--19) containing the whole active domain and an arbitrarily chosen head element 
from the cluster (cf. Section~\ref{sec:meth_cbfr}) and then identifies the minimal similarity inside the cluster between all terms from the active domain 
and the cluster head (line~22--27). In the following, there are created new clusters with new head elements based on the minimal similarity of a term to the 
head of a cluster where it belongs to as long as the similarity threshold is not exceeded and all terms are reassigned if they are more similar to the head
of the new created cluster (line~30--94). The procedure iterates as long as there are still elements inside one of the clusters that have a similarity to the
corresponding head element that is lower than the similarity treshold alpha (while-condition in line~31). Hence, the iteration proceeds until each element of 
the active domain is clustered such that the minimal similarity with regards to the threshold $\alpha$ from Definition~\ref{def:cbfr} can be ensured. The 
obtained clustering is finally returned and used later to induce a horizontal fragmentation (cf. Definition~\ref{def:cbfr}).
The following example shows how a clustering is computed for a given set of disease terms and the pairwise similarities of the diseases.

\begin{exmp}
\label{sec:impl_clust_exmp}

Consider the same sample disease term set as already examined in Example~4 in \cite{Wiese2014},
\begin{align*}
    \{Asthma, Cough, Influenza, Tibial Fracture, Ulna Fracture\},
\end{align*}
and the corresponding pairwise similarities of the diseases like shown in Table~\ref{tab:impl_similarities} in order to illustrate the functionality of the 
clustering procedure. All the respiratory diseases are similar to one another and the same holds for the two fractures but fractures are obviously not that
similar to respiratory diseases, and vice versa.

\begin{table}[h]
    \centering
    \begin{tabular}{|c|c c c c c|}
        \hline
         & Asthma & Cough & Influenza & Tibial Fracture & Ulna Fracture \\
        \hline
        Asthma & 1 & 0.2 & 0.2 & 0.1429 & 0.1429 \\
        Cough & 0.2 & 1 & 0.2 & 0.1429 & 0.1429 \\
        Influenza & 0.2 & 0.2 & 1 & 0.1429 & 0.1429 \\
        Tibial Fracture & 0.1429 & 0.1429 & 0.1429 & 1 & 0.3333 \\
        Ulna Fracture & 0.1429 & 0.1429 & 0.1429 & 0.3333 & 1 \\
        \hline
    \end{tabular}
    \caption{Pairwise similarities of some diseases (obtained via UMLS::Similarity)}
    \label{tab:impl_similarities}
\end{table}

Let $\alpha=0.2$. The clustering procedure starts by creating the first cluster $c_1$ that is represented by an arbitrarily chosen head, e.g. the first disease 
of the set ($head_1="Asthma"$), and moves all remaining terms to the cluster's subset. In the next step, the minimal similarity of any disease term to the 
cluster head, $s_{min}=\min\limits_{a \in c_1}~sim(a, head_1)$, is calculated and compared to the similarity threshold. As the minimal intracluster similarity
still is below the threshold, $s_{min}=0.1429 < 0.2=\alpha$, the identified minimizing disease term is chosen to be the head element of the next cluster, i.e. 
$\argmin\limits_{a \in c_1}~sim(a, head_1)="Tibial Fracture"=head_2 \in c_2$. After this step, all diseases of cluster $c_1$ that are more similar to the
fracture, that is, the other fracture disease, than to asthma will be reassigned to the cluster $c_2$. The clustering computation is complete as the minimal 
intracluster similarities for all clusters are above the thresold: 
\begin{align*}
    \forall c_i, i \in \{1,2\}: \min\limits_{a \in c_i}~sim(a, head_i) \geq \alpha
\end{align*}
The resulting clustering is $\{c_1, c_2\}$ where $c_1=\{Asthma,Cough,Influenza\}$ and $c_2=\{Tibial Fracture, Ulna Fracture\}$. If the similarity threshold is
below the minimal pairwise similarity of any two diseases, e.g. $\alpha=0.14$, then the computation stops after the initial step before the first iteration
and the resulting clustering consists of only one cluster. If $\alpha > 0.2$, there would be one cluster with the two fractures and a own cluster for each of 
the respiratory diseases.

\end{exmp}



\subsubsection{Scaling parameter $\alpha$}
\label{sec:impl_clust_alpha}

The similarity threshold $\alpha$ which is used in the computation of the clustering of the active domain of the relaxation attribute (\verb!Disease!) 
determines how big the similarity of a term to the head of a cluster has to be at least such that they will be assigned to the same cluster 
(cf. Definition~\ref{def:cbfr}). By choosing an appropriate threshold, it is assured that no two diseases, represented by terms, that have no significant
similarity at all, belong to the same cluster and are subsequently seen as similar diseases. Of course, the chosen similarity measure is rather simple and the
threshold was chosen arbitrarily as not the full MeSH taxonomy, i.e. all disease terms in this case, could be considered and only the simplest measure based 
solely on the path length of the shortest path between two diseases was used to obtain the similarity values. Appendix~\ref{app:alpha} shows how the number of 
clusters and also the average number of terms per cluster for the different term subsets (10, 30 or 100 terms) varies when scaling the similarity threshold
parameter. Like shown, an alpha value of at least $0.2$ always caused the clustering to contain a lot of clusters whereas the corresponding active domain 
subsets of each cluster often only contained a head element, thus, a that big value is not reasonable and far from the idea behind a clustering that clusters
similar diseases and allows for similarity-based and flexible answering. For $\alpha=0.1$, the two smallest term subset lead to a clustering consisting of only
one cluster with all the terms in it (cf. Appendix~\ref{app:alpha_10terms}, \ref{app:alpha_30terms}), which indicates that for these two subsets the minimal
similarity of two diseases was higher than $0.1$, and the biggest term subset lead to 3 big clusters. In conclusion, the chosen value for $\alpha$ should be
between $0.1$ and $0.2$ as other values would cause an unreasonable clustering and for the further evaluations the chosen value for alpha was fixed for the
different term subsets, i.e. for 10 and 100 terms $\alpha=0.12$ and for 30 terms $\alpha=0.15$.


\subsection{Ignite Reference Implementation}
\label{sec:impl_refimpl}
The here described approach is used as reference when comparing the other approaches regarding the query execution time. In the intended way, this
implementation of the simple medical information system in a distributed in-memory \citetalias{Ignite} database is achieved by creating two partitioned tables
to the corresponding relations in the database. These two tables are collocated via their shared attribute, the personal ID, such that personal information and
the diseases a person suffers from are stored together. The corresponding SQL data definition langauge (DDL) statements are

\begin{verbatim}
    CREATE TABLE INFO (
        ID INT PRIMARY KEY,
        Name VARCHAR,
        Address VARCHAR,
        Age INT
    ) WITH "template=partitioned,backups=0,affinityKey=ID"
\end{verbatim}
for the patients personal information and
\begin{verbatim}
    CREATE TABLE ILL (
        ID INT,
        Disease VARCHAR,
        MeshID VARCHAR,
        PRIMARY KEY (ID, Disease)
    ) WITH "template=partitioned,backups=0,affinityKey=ID"
\end{verbatim}
for the listing of all the diseases for each patient. Both tables are partitioned, i.e. fragmented, horizontally by an abstract assignment of tuples to
partitions based on a hash function applied to their affinity keys (attribute \verb!ID!) to ensure the collocation of the data. The data is partitioned and 
distributed in an arbitrary way where only collocation via the attribute \verb!ID! is guaranteed but not that any similar disease terms are collocated because
this approach is implemented without the clustering-based fragmentation which disables the possibility to make use of efficient similarity-based query
answering. Furthermore, flexible query answering can not be combined with the similarity-based query answering (cf. Section~\ref{sec:meth_fqa_fqsba}) either,
thus, it requires for some less intelligent and less efficient query rewriting where the generalization of the query is achieved as disjunction that covers all
similar constant symbols as shown in Example~\ref{sec:meth_fqa_exmp}. Only the collocation of the two relations via their shared attribute, which can be 
modeled as foreign key constraints in other relational database systems, allows the DDBS to gain some efficiency as costly data transfer between the nodes
(servers) of the cluster can be avoided if the data to be joined is available locally.

\subsection{Implementation Alternatives}
\label{sec:impl_alter}

This section describes the two developed implementation alternatives that utilize the by \citetalias{Ignite} provided technical infrastructure together with a
clustering-based fragmentation based on the similarity metric defined for the MeSH disease terms in order to form a medical information system for patient data 
which is distributed across the cloud servers according to the horizontal fragmentation. The query processing in this two approaches, especially in terms of 
similarity-based and flexible answering, is discussed in detail in Section~\ref{sec:impl_qpro}.

\subsubsection{Materialized Fragment Approach}
\label{sec:impl_alter_mater}
The first approach is the "materialized fragment" approach that stores the horizontal fragments, that are induced by the precomputed clustering of the disease
terms based on the similarity metric, as separate relations in the database with different table names for the different fragments. Although the materialized
fragments in the examples have names that cover the content represented by the corresponding cluster, e.g. the semantical characterization of the different
products through the fragment name as in Example~\ref{sec:meth_fqa_exmp2}, the fragment table names in this approach are only generated syntactically, for the
sake of simplicity, by concatenation of the underlying relation name, an underscore and the fragment number, e.g. \verb!ILL_0!. To achieve the distribution of
the horizontal fragments, that are now separate tables and not longer belong together syntactically from the DDBS's point of view, to the servers, a modified
affinity function has to be explicitly provided and made available on all nodes. This affinity function handles the mapping of the materialized fragments to 
the servers and also provides the similarity function for two disease terms that is required for both the clustering and the query rewriting and answering 
(cf. Section~\ref{sec:impl_qpro_sbfq}). Furthermore, the collocation of the \verb!ILL! and \verb!INFO! relation fragments via a derived horizontal 
fragmentation can only be achieved by storing the personal information of patients (\verb!INFO!-tuples) at each server that hosts also information about a
disease that this patient has, i.e. to each \verb!ILL! fragment there is also an \verb!INFO! fragment present on the same server for appropriate collocation.
Still, the horizontal fragmentation (cf. Section~\ref{sec:theo_ddb_frag}) and the materialization of the fragments of the relations as separate relations is 
still transparent for the user as the clustering and the because of that required rewriting of queries is done automatically by the system. For example, if a
patient has a disease $x$ that belongs to cluster $c_j$, then the patient's personal information will be stored in the relation \verb!INFO_j! that is collocated
to the relation \verb!ILL_j! containing the tuple with the reference to the patient and information about disease $x$. It is possible that a patient has multiple 
diseases that not necessarily belong to the same cluster and due to this the personal information of the patient may be stored redundantly in multiple 
materialized fragments, too.


\subsubsection{Partition Number Approach}
\label{sec:impl_alter_partnum}

The second approach, the "partition number" approach, also materializes the fragments analogously to the first approach but in an from the SQL perspective
invisible way by assigning different partition numbers to the different fragments of the relations via the affinity function API in the intended way instead of
different relation names. The partitions of the partitioned relations are then distributed and mapped to the servers. By doing so, the queries do not require to
be rewritten as in the first approach because the query processing is almost completely done by Ignite's query processor as the relation names are not different
for each fragment like in the other approach but the different fragments are distinguished by their partition numbers. Additionally, the fragmentation is again
fully transparent for the user and hidden from SQL. Collocation (cf. Section~\ref{sec:meth_ign_affc}) of the \verb!INFO! partitions to the \verb!ILL! partitions 
is also achieved with the help of the affinity function and the identification of the correct partition number that is inferred from the clustering, i.e. if a
patient has disease $x$ that belongs to cluster $c_j$, then partition $j$ of the \verb!ILL! relation stores the tuple stating that this patient has disease $x$ 
and additionally partition $j$ of the \verb!INFO! relation is responsible for the tuple with the patient's personal information. Both partitions are hosted by 
the same server due to collocation, and personal patient data may be stored redundantly in multiple partitions. To guarantee collocation, this redundancy can 
not be avoided for the derived fragmentation (cf. Section~\ref{sec:theo_ddb_frag}) in both approaches.



\subsection{Query processing}
\label{sec:impl_qpro}

Most of the query processing and execution is left up to \citetalias{Ignite} itself as it incorporates a parser as well as distribution and optimization
mechanisms for SQL queries. Especially, if data is collocated and can be executed in a collocated fashion instead of a non-collocated execution, the cluster 
can benefit from the collocation and benefit from the application of optimizations with more efficient query answering. However, in order to bring the
intelligence into the application by utilizing the knowledge about the clustering-based fragmentation (Section~\ref{sec:meth_cbfr}) of the stored data, 
additional query rewriting and redirection are both required and helpful when trying to avoid costly distributed joins (cf. Section~\ref{sec:meth_ign_affc})
across all nodes in the cluster and further optimize the execution with the help of similarity-based query answering (cf. Section~\ref{sec:meth_sbqa}).

\subsubsection{Rewriting}
\label{sec:impl_qpro_rewr}

JSqlParser is an open source Java SQL parser tool (provided at \url{https://github.com/JSQLParser/JSqlParser}) that translates SQL queries in a hierarchy 
of java classes that can be navigated with the so-called visitor design pattern \citep[pp.~331ff.]{Gamma1994} and allows for identifying and modifying all the
subexpressions in the query, e.g. comparison expressions contained in the \verb!WHERE! clause of the query. This tool is used in the implementation 
to rewrite all the queries according to the underlying clustering-based fragmentation. In particular, this is implemented by parsing the given string that 
represents an SQL query, traversing the generated Java class hierarchy, modifiying all the classes that belong to expressions referring to the relations in
the database, e.g. a selection condition or any \verb!FROM! expression, and returning a string that represents the rewritten query. Besides, this also allows
for interoperability of the rewriting mechanisms for any other SQL backend. 

\begin{exmp}
\label{sec:impl_qpro_exmp}
For instance, assume the query
\begin{verbatim}
    SELECT * FROM ILL WHERE ILL.disease = 'Cough'
\end{verbatim}
and a clustering-based fragmentation as described in Example~\ref{sec:impl_clust_exmp} (for $\alpha=0.2$). The parsing of the query string results in a 
traversable class hierarchy that also contains a comparison expression class, i.e. the comparison of the column reference \verb!ILL.disease! to the constant 
symbol \verb!'Cough'!, a disease term. Then, by calculating the similarity of the disease in the selection condition, a cluster with maximal similarity is
identified, thus, the query can be rewritten to match the fragment that corresponds to the identified cluster. This is done by modifiying the in the 
\verb!FROM! clause contained table expression, \verb!FROM ILL!, to reference the relevant fragment of the table:
\begin{verbatim}
    SELECT * FROM ILL_0 WHERE ILL.disease = 'Cough'
\end{verbatim}
\end{exmp}

The reference implementation system is not able to handle a query in a way like described in Example~\ref{sec:impl_qpro_exmp} as the fragmentation of the data
is arbitrarily computed and completely apart from a clustering-based fragmentation (cf. Section~\ref{sec:meth_cbfr}). This query can only be sent
to all the servers in a MapReduce \citep{MapReduce2004} fashion where the query is executed locally and in parallel on all caching nodes (mapping step) and the 
results are accumulated as a union of all result sets into a single result set (reduce step). The other two approaches (Section~\ref{sec:impl_alter}), on the
contrary, are able to identify the server that is responsible for the only relevant fragment of the data and process the queries in their own, more efficient
way (cf. Section~\ref{sec:impl_qpro_sbfq}).


\subsubsection{Similarity-based and Flexible Query Answering}
\label{sec:impl_qpro_sbfq}

The here developed approaches are, as already mentioned before, able to rewrite given SQL queries such that they match the underlying clustering-based
fragmentation, which clusters the data with the similarity of the diseases. In case of the first approach (Section~\ref{sec:impl_alter_mater}), where the 
fragments of the relations \verb!ILL! and
\verb!INFO! are materialized as separate SQL tables, it is required to rewrite queries according to that materialization, i.e. all the table expressions in the
queries have to be adapted to the corresponding fragment. If there is a selection condition on the relaxation attribute in the query for a certain disease, the
single relevant fragment is determined and rewritten. If there is no such selection condition, a localization program has to be used in order to answer the 
query correctly by considering all existing fragments. Another execution strategy here is again MapReduce \citep{MapReduce2004} which would first rewrite the 
query once for each fragment to obtain a fragment-specific subquery, and then map each subquery to only the server hosting that certain fragment. This strategy
only works for not too complex queries as it relies on the improved data locality of the collocation. For instance, queries requiring for inter-fragment joins,
i.e. queries that do not join the relations \verb!ILL! and \verb!INFO! via their collocating attribute (identificators of the persons) or queries that join a
relation with itself like the last query of Appendix~\ref{app:queries}, are computationally highly expensive as they have to be rewritten for all combinations
of any two or even more fragments. This results in a combinatorial explosion of the number of subqueries in both MapReduce and the localization program 
(cf. Section~\ref{sec:theo_ddb_frag}) which causes bad performance at the query execution.

In case of the second approach (Section~\ref{sec:impl_alter_partnum}), where fragments of the relations are hidden from the SQL perspective as the fragmentation
is done by assigning a partition
number to each fragment and mapping each partition to one of the servers, query rewriting is not required because the partitions are known to Ignite's query
processor already and the query contains the correct relation names. Nevertheless, as again selection conditions on the relaxation attribute allow to answer
queries based on the disease similarity metric and the clustering (Section~\ref{sec:meth_cbfr}), the DDBS is capable of executing queries only on the server 
that hosts the corresponding fragment or partition by identifying the most similar cluster first and then using Ignite's SQL API (see
\url{https://apacheignite-sql.readme.io/docs/java-sql-api} for details) in order to set the partition number for the query according to the identified 
fragment. For example, a similarity-based rewriting and answering of the query in Example~\ref{sec:impl_qpro_exmp} as for the first approach is not necessary
for the second approach as only the partition number has to be identified and set via the appropriate class method of the SqlQuery or SqlFieldsQuery of 
Ignite's SQL API (see \url{https://apacheignite-sql.readme.io/docs/java-sql-api}). 


With the disease term similarity, flexible query answering by generalization (cf. Section~\ref{sec:meth_fqa_qgen}) of the selection condition on the relaxation 
attribute can be performed
as query rewriting, too, for all the implementations. The reference implementation requires for substituting the comparison of the disease attribute to a 
single constant symbol (a disease term) with a set of constant symbols (other possible disease terms). The ways how this can be done were also shown previously 
in Example~\ref{sec:meth_fqa_exmp} where the generalized query with the set of constant symbols, which can be expressed in FOL as a disjunctive formula, was
translated into SQL, e.g. with the help of an SQL \verb!IN!-clause. Due to the fact that the clustering and the query generalization fit well together
(cf. Section~\ref{sec:meth_fqa_fqsba}), the two
developed approaches do not need a substitution of the selection condition with an SQL equivalent of the disjunctive FOL formula because the fragments or
partitions inherently cover the more general sets of constant symbols that are equal to the corresponding clusters. This means that -- instead of substituting
the selection condition by another expression -- it is simply omitted, and the query is adapted analogously to the similarity-based answering by restricting it 
to that certain fragment that belongs to the cluster with the more general diseases (incl. the initial one) such that then all answers only need to be obtained
from this fragment (or partition). Thus, for the materialized fragment approach (Section~\ref{sec:impl_alter_mater}) only the relation names have to changed to 
the corresponding fragment whereas
for the other approach (Section~\ref{sec:impl_alter_partnum}) again only the corresponding partition number has to be set via the SQL API. 

The interaction of similarity-based and flexible query answering was also part of Example~\ref{sec:meth_fqa_exmp2} demonstrating the well-suitedness of both
techniques as the query was rewritten to yield all answers solely from the aircraft fragment which on the one hand generalizes the query and on the other hand
offers for efficient similarity-based query execution. Furthermore, overgeneralization \citep{Wiese2014} is avoided as the disease attribute of the obtained 
answers is inherently restricted to terms of the same cluster providing a semantically close result set instead of a result set that also includes too general
answers (cf. Section~\ref{sec:meth_fqa_anti}).

