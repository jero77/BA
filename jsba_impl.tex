%!TEX root = jsba_main.tex
% Implementation

\section{Implementation}
\label{sec:impl}

\todo[inline]{Summarize following subsections, probably also intro to medical information system (e.g. how, simplif., data, relations, security, etc.)}

\subsection{Ignite Reference Implementation}
\label{sec:impl_refimpl}
In the intended way, the implementation of the simple medical information system in a distributed in-memory \citetalias{Ignite} database is achieved by 
creating two partitioned tables to the corresponding relations in the database. These two tables are collocated via their shared attribute, the personal ID,
such that personal information and the diseases a person suffers from are stored together. The corresponding SQL data definition langauge (DDL) statements are

\begin{verbatim}
    CREATE TABLE INFO (
        ID INT PRIMARY KEY,
        Name VARCHAR,
        Address VARCHAR,
        Age INT
    ) WITH "template=partitioned,backups=0,affinityKey=ID"
\end{verbatim}
for the patients personal information and
\begin{verbatim}
    CREATE TABLE ILL (
        ID INT,
        Disease VARCHAR,
        MeshID VARCHAR,
        PRIMARY KEY (ID, Disease)
    ) WITH "template=partitioned,backups=0,affinityKey=ID"
\end{verbatim}
for the listing of all the diseases for each patient. Both tables are partitioned, i.e. fragmented, horizontally by an abstract assignment of tuples to
partitions based on a hash function applied to their affinity keys (attribute \verb!ID!) to ensure the collocation of the data. This approach is implemented
without the clustering-based fragmentation which disables the possibility to make use of efficient similarity-based query answering. Furthermore, flexible 
query answering can not be combined with the similarity-based query answering (cf. Section~\ref{sec:meth_fqa_fqsba}), either, thus, it requires for some less
intelligent and less efficient query rewriting where the generalization of the query is achieved as disjunction that covers all similar constant symbols as
shown in Example~\ref{sec:meth_fqa_exmp}. Only the collocation of the two relations via their shared attribute, which can be modeled as foreign key constraints
in other relational database systems, allows the DDBS to gain some efficiency as costly data transfer between the nodes (servers) of the cluster can be avoided
if the data to be joined is available locally.

\subsection{Implementation Alternatives}
\label{sec:impl_alter}

\subsection{Clustering}
\label{sec:impl_clust}

\subsubsection{Scaling parameter alpha}

\subsection{Query processing}
\label{sec:impl_qpro}

\subsubsection{Rewriting}

\subsubsection{Similarity-based and Flexible Query Answering}