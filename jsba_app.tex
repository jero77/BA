%!TEX root = jsba_main.tex

% Appendix
\section{Appendix}
\label{app}

% List of Queries
\subsection{Sample Queries}
\label{app:queries}

\begin{itemize}

    \item[$Q_1$:] \begin{verbatim}
        SELECT avg(p.age)
        FROM INFO p
    \end{verbatim}

    \item[$Q_2$:] \begin{verbatim}
        SELECT p.name, p.age, p.address 
        FROM ILL i, INFO p 
        WHERE i.id = p.id 
            AND i.disease='Liver Failure'
    \end{verbatim}
    
    \item[$Q_3$:] \begin{verbatim}
        SELECT p.name, p.age, p.address 
        FROM ILL i1, ILL i2, INFO p 
        WHERE i1.id = p.id 
            AND i2.id = p.id 
            AND i1.disease='Liver Failure' 
            AND i2.disease='Hemoptysis'
    \end{verbatim}

    \item[$Q_4$:] \begin{verbatim}
        SELECT i.disease, p.* 
        FROM ILL i, INFO p 
        WHERE i.id = p.id
    \end{verbatim}
   
    \item[$Q_5$:] \begin{verbatim}
        SELECT p.name, count(i.disease) 
        FROM ILL i, INFO p 
        WHERE i.id = p.id 
        GROUP BY p.name 
    \end{verbatim}

    \item[$Q_6$:] \begin{verbatim}
        SELECT p.name, i1.disease, i2.disease 
        FROM ILL i1, ILL i2, INFO p 
        WHERE i1.id = p.id 
            AND i2.id = p.id 
            AND i1.disease <> i2.disease
    \end{verbatim}


\end{itemize}
\newpage

\subsection{Sample Queries for Flexible Query Answering}
\label{app:flexqueries}

\begin{itemize}

    \item[$Q_7$:] \begin{verbatim}
        SELECT p.* 
        FROM INFO p, ILL i 
        WHERE p.id = i.id 
            AND i.disease = 'Liver Failure'
    \end{verbatim}
    
    \item[$Q_8$:] \begin{verbatim}
        SELECT p.name, p.age 
        FROM INFO p, ILL i 
        WHERE p.id = i.id 
            AND i.disease = 'Liver Failure'
            AND p.age < 50
    \end{verbatim}
    

\end{itemize}



% MeSH Terms and SPARQL query
\subsection{MeSH Terms}
\label{app:terms}

This following terms are a randomly chosen subset of all disease describing Medical Subject Headings 2019 terms obtained from the SPARQL endpoint 
(that can be found under \url{http://id.nlm.nih.gov/mesh/sparql}). The corresponding SPARQL query is
\begin{verbatim}
    PREFIX rdf: <http://www.w3.org/1999/02/22-rdf-syntax-ns#>
    PREFIX rdfs: <http://www.w3.org/2000/01/rdf-schema#>
    PREFIX owl: <http://www.w3.org/2002/07/owl#>
    PREFIX meshv: <http://id.nlm.nih.gov/mesh/vocab#>
    SELECT DISTINCT ?N
    FROM <http://id.nlm.nih.gov/mesh>
    WHERE { 
        ?D a meshv:Descriptor .
        ?D rdfs:label ?N .
        ?D meshv:treeNumber ?TN .
        FILTER(REGEX(?TN,'C'))
    }
\end{verbatim}
which queries for all disease names (\verb!rdfs:label!) from all descriptors contained by filtering out only those descriptors that are contained in the
tree identified by 'C', the disease tree, with the help of the filter \verb!FILTER(REGEX(?TN,'C'))!.

\paragraph{Term set (semicolon-separated):}$ $\\
Choroidal Neovascularization; Keratitis; Catalepsy; Liver Diseases, Alcoholic; Liver Failure; Iliac Aneurysm; Hemoptysis; Fallopian Tube Neoplasms;
Paralysis, Obstetric; Pre-Excitation, Mahaim-Type; Setariasis; Neurodegenerative Diseases; Hepatitis B, Chronic; Vaginal Discharge; Hypogonadism;
Spinal Cord Vascular Diseases; Abnormalities, Drug-Induced; Central Nervous System Viral Diseases; Soft Tissue Neoplasms; Collagen Diseases; 
Tuberculosis, Female Genital; Contrecoup Injury; Brugada Syndrome; Echinococcosis; Skin Manifestations; Multiple Endocrine Neoplasia Type 1; 
Focal Infection; Angiokeratoma; Nose Diseases; Bacteroidaceae Infections; Dental Fissures; Intraoperative Awareness; Facial Neoplasms; Trypanosomiasis;
Latent Tuberculosis; Congenital Hyperinsulinism; Alzheimer Disease; Ascorbic Acid Deficiency; Serratia Infections; Morning Sickness; 
Amyloid Neuropathies, Familial; Methemoglobinemia; Abdominal Neoplasms; Tuberculosis, Lymph Node; Mucopolysaccharidosis III; Blackwater Fever; Hot Flashes;
Adrenal Rest Tumor; Squamous Intraepithelial Lesions of the Cervix; Ichthyosis Bullosa of Siemens; Sleep Paralysis; Disease Susceptibility; 
HIV Seropositivity; Cowpox; Spermatocele; Gait Apraxia; Miliaria; Focal Epithelial Hyperplasia; Adrenal Insufficiency; Putaminal Hemorrhage; 
Forearm Injuries; Multiple Endocrine Neoplasia Type 2b; Burkholderia Infections; Brain Concussion; Compassion Fatigue; Rat-Bite Fever; Bone Malalignment;
Equinus Deformity; Pure Autonomic Failure; Chediak-Higashi Syndrome; Anemia, Diamond-Blackfan; Confusion; Herpes Simplex; Migraine without Aura;
Kwashiorkor; Eye Abnormalities; Esophagitis, Peptic; Hepatitis, Viral, Animal; Lip Diseases; Granuloma, Foreign-Body; Papilloma, Inverted; 
Mediastinal Emphysem; Bell Palsy; Port-Wine Stain; Intellectual Disability; Lingual Goiter; Gynecomastia; Alternariosis; Sinoatrial Block; Agranulocytosis;
Intestinal Diseases; Gingivitis, Necrotizing Ulcerative; Trichuriasis; Tachycardia; Erythema Infectiosum; Impetigo; Pyomyositis; Hematocolpos; Cough;
Mental Retardation, X-Linked


\newpage
\subsection{Clustering - Parameter $\alpha$}
\label{app:alpha}

\subsubsection{10 Terms}
\label{app:alpha_10terms}
\begin{table}[h]
    \centering
    \begin{tabular}{|c|c|c|}
        \hline
        $\alpha$ & Clusters & Terms/Cluster  \\
        \hline
        0.2 & 9 & 1.$\overline{1}$ \\
        0.15 & 7 & 1.$\overline{428571}$ \\
        0.13 & 6 & 1.$\overline{6}$ \\
        0.12 & 5 & 2 \\
        0.1 & 1 & 10 \\
        \hline
    \end{tabular}
    \caption{Number of clusters and cluster size for different values for threshold parameter $\alpha$ restricted to the first 10 terms of the
            term set (Appendix~\ref{app:terms})}
    \label{tab:app_10terms}
\end{table}


\subsubsection{30 Terms}
\label{app:alpha_30terms}
\begin{table}[h]
    \centering
    \begin{tabular}{|c|c|c|}
        \hline
        $\alpha$ & Clusters & Terms/Cluster  \\
        \hline
        0.2 & 19 & $\approx$1.58 \\
        0.15 & 12 & 2.5 \\
        0.14 & 9 & 3.$\overline{3}$ \\
        0.13 & 9 & 3.$\overline{3}$ \\
        0.12 & 8 & 3.75 \\
        0.11 & 2 & 15 \\
        0.1 & 1 & 30 \\
        \hline
    \end{tabular}
    \caption{Number of clusters and cluster size for different values for threshold parameter $\alpha$ restricted to the first 30 terms of the
            term set (Appendix~\ref{app:terms})}
    \label{tab:app_30terms}
\end{table}


\subsubsection{100 Terms}
\label{app:alpha_100terms}
\begin{table}[h]
    \centering
    \begin{tabular}{|c|c|c|}
        \hline
        $\alpha$ & Clusters & Terms/Cluster  \\
        \hline
        0.2 & 51 & $\approx$2 \\
        0.15 & 34 & $\approx$2.94 \\
        0.13 & 22 & 4.$\overline{54}$ \\
        0.12 & 15 & 6.$\overline{6}$ \\
        0.11 & 6 & 16.$\overline{6}$ \\
        0.1 & 3 & 33.$\overline{3}$ \\
        \hline
    \end{tabular}
    \caption{Number of clusters and cluster size for different values for threshold parameter $\alpha$ for all 100 terms of the term set
            (Appendix~\ref{app:terms})}
    \label{tab:app_100terms}
\end{table}


% Code listings
\newpage
\subsection{Clustering Java Code}
\label{app:java}
\lstdefinestyle{javacode}{
    language=Java,
    basicstyle=\tiny,
    numbers=left,
    numbersep=2pt,
    numberstyle=\tiny, 
    showspaces=false, 
    showstringspaces=false
}

\subsubsection{Cluster Class}
\label{app:java_cluster}
\lstinputlisting[style=javacode]{sourcecode/Cluster.java}


\subsubsection{Clustering Procedure}
\label{app:java_clustering}
\lstinputlisting[style=javacode, firstline=280, lastline=376]{sourcecode/ClusteringAffinityFunction.java}
