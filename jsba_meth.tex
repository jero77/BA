 %!TEX root = jsba_main.tex
% Method section

\section{Methods}
\label{sec:meth}
 
 \begin{enumerate}
     \item show how your choice of design and research method is suited to answering your research question(s)
     \item demonstrate that you have given due consideration to the validity and reliability of your chosen method
     \item by “showing” instead of “telling”, you demonstrate that you have understood the practical meaning of these concepts
     \item Show the reader what you have done in your study, and explain why. 
     \begin{enumerate}
         \item How did you collect the data? 
         \item Which options became available through your chosen approach?
         \item What were your working conditions? 
         \item What considerations did you have to balance?
     \end{enumerate}
    \item Tell the reader what you did to increase the validity of your research. 
    \begin{enumerate}
        \item E.g., what can you say about the reliability in data collection? 
        \item How do you know that you have actually investigated what you intended to investigate? 
        \item What conclusions can be drawn on this basis? 
        \item Which conclusions are certain and which are more tentative?
        \item Can your results be applied in other areas? Can you generalise? If so, why? If not, why not?
    \end{enumerate}
    \item You should aim to describe weaknesses as well as strengths. An excellent thesis distinguishes itself by defending – and at the same time criticising – the choices made.
 \end{enumerate}
 
 
 \subsection{Ignite}
 
 \subsection{Clustering-based Fragmentation}
 
 \subsection{Similarity-based Query Answering}