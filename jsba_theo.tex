%!TEX root = jsba_main.tex
% Theoretical background 

\section{Theoretical Background}
\label{sec:theo}

\subsection{Distributed Databases}
\label{subs:ddb}

A \emph{distributed database}~(DDB) can be defined as multiple, logically interconnected databases spread across a computer network~\cite[p.~4]{Ozsu1991}, such that the physical presence of the data may be dispersed spatially instead of having one single database hoarding all the data at a certain location physically. 
On top of this DDB, there is often a so-called \emph{distributed database management system}~(DDBMS) which manages the underlying databases, and the users are 
enabled to access and work with the DDB without noticing the physically distribution, i.e. in a transparent manner~\citep{Ozsu1991}. This results, from the point
of view of a user, in a logically single database, which can be queried in the same way as a non-distributed database. Furthermore, users can benefit from a
physical distribution of the data depending on an architectural design because they can communicate with the geographically closest site where one of the 
databases resides reducing the network communication delay, and concurrent access of the DDB can be handled in a distributed way, too, exploiting the fact that 
more servers imply more computing power for the \emph{distributed database system} (DDBS).

A DDBMS can be \emph{homogenous} or \emph{heterogenous} where in a homogenous DDBMS all participating sites share the same DBMS locally and in a heterogenous
DDBMS the sites may have a different DBMS with probably different data models.