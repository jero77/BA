%!TEX root = jsba_main.tex
% Introduction

\section{Introduction}
\label{sec:intro}

\emph{Distributed databases} (DDBs) gained significantly more importance over the past years as they provide physically distributed storage \citep{Tan2009}
for the ever-growing amount of data in our world. Especially, in fields where the amount of data to be hosted in a database is challenging, e.g. multimedia
and textual data occurring in a social network with millions of people or medical so-called "big data" retrieved and collected from different sources to be
used in healthcare for e.g. predictions or research as mentioned in \citet{Lee2017}, a fragmentation and replication of this big data to multiple servers 
in a cloud storage infrastructure instead of one single server can pay off. The data fragmentation overcomes the limitation of storage capacity of a single
storage instance as the total amount of data can be split into smaller parts, so-called fragments, and fragmentation can increase the performance of the 
query processing. 
Furthermore, the replication of the data fragments inside the network yields a tolerance to failures by compensating data loss with recovery, e.g. in 
case of a malfunction or total failure of a single server, to guarantee the desired availability and reliability of the data accessed by the users of the
distributed database system (DDBS).


In the case of a relational data model underlying the database, a fragmentation of the data can be a \emph{horizontal} or \emph{vertical fragmentation} 
of the relations (tables) of the database \citep[p.~104f.]{Ozsu1991}. A horizontal fragmentation means a tuple-wise assignment to fragments, i.e. fragments
contain subsets of rows of the non-fragmented table, whereas a vertical fragmentation "slices" the relation by its attributes, i.e. fragments contain
projections to the columns. 
The here analyzed and implemented approach uses a clustering heuristic on one chosen attribute as described in \citet{Wiese2014} to obtain a primary
horizontal fragmentation for a relation. The clustering procedure depends on a similarity calculation of the values of the chosen relaxation attribute 
resulting in a clustering of the column's values, such that values with a high similarity belong to the same cluster. Hence, a fragmentation of relations
that have common attributes with this horizontally fragmented relation, i.e. attributes that will be used in queries to join these two relations, can be 
\emph{derived}~\citep[p.~116f.]{Ozsu1991}, such that the tuples which are about to be joined reside on the same server allowing for better performance in 
queries compared to query processing with a distributed join which has to be applied if the data is not residing at the same
site~\citep{Sattler2009DistJoin}.
Thus, this derived horizontal fragmentation improves data locality~\citep{Wiese2014} because the probably high delay of data transmission over the 
network, which is needed in case of a distributed join, will be avoided to the joy of any database user.


\defcitealias{MeSH}{MeSH\textcopyright}
The example of a \emph{medical information system}, which is used as an application of such a distributed database system with a clustering-based 
fragmentation, is a modified version of the example defined in \citet{Wiese2014}. Patient's personal information as well as the diseases they suffer from
are contained in a distributed database. For example, a hospital could host this database to provide access to the patient data to doctors and nurses that
need to work with this data from different computers or other technical devices located in the hospital, but it is also possible that researchers could
make use of that patient data, e.g. when investigating the co-occurrence of several diseases.
The clustering procedure is applied to this disease information of patients, whereas the disease terms are conform to the vocabulary provided by the 
\emph{Medical Subject Headings} taxonomy (see \citetalias{MeSH}) of the U.S. National Library of Medicine, which is the largest biomedical library in 
the world. The \emph{Unified Medical Language System} (UMLS), which was also developed and maintained by the U.S. National Library of Medicine, brings
together several taxonomies, like the MeSH taxonomy, in order to integrate and link their biomedical and health-related vocabularies allowing applications
to interchange information based on different vocabularies.
A similarity measure defined between the disease terms from the MeSH database yields the needed similarity calculation for the clustering-based 
fragmentation. In this way, tuples holding information about patients suffering from similar illnesses are stored on the same site and can easily be
queried together in an intelligent, non-distributed fashion.


Upon query failure, i.e. when the database cannot give an exact answer to a query formulated by a user, an empty result is returned. Assuming the query 
was formulated correctly, the blank result indicates to the user that the information he or she was looking for is not present in the current database
state due to the lack of any exact answers, and so this result is non-informative for the user in the general case. \emph{Flexible query answering}, on 
the contrary, can overcome this lack of information when the database system uses techniques like \emph{query generalization} to provide the user a 
non-empty result, which on the one hand does not match the query exactly but on the other hand contains information instead that may be relevant to the
user.

\begin{exmp}
\label{sec:intro_exmp}
Consider a sample database about companies and their employees. The database contains a relation 'Employee' with information about a person, e.g. the
person's name, age, address, salary, social security number and a reference to the company the person works for. The companies are stored in the
relation 'Company' that has the attributes name, headquarter, industrial sector and maybe some others. An exemplary SQL query that joins the two relations
to find out the name of those companies in a certain business where the average salary of all employees is higher than 100.000~\$ could be
\begin{verbatim}
    SELECT Company.name 
    FROM Employee, Company 
    WHERE Employee.Company = Company.Name 
        AND Company.Industry = 'Farming'
    GROUP BY Company.Name 
    HAVING avg(salary) > 100000.
\end{verbatim}
However, as the database maybe does not know a company from that agricultural sub-sector having an average salary like that, it could return an empty 
answer to the query that is not satisfying for the user. Instead of returning and empty answer set, the database could - of course only if it is able 
to do so - try to answer the given query in a flexible manner, e.g. by changing the selection condition in the \verb!WHERE! clause
\begin{verbatim}
    Company.Industry = 'Farming'
\end{verbatim}
to a more \emph{general} one that could provide some information not matching the query exactly but still having some informational content. One way to 
achieve this, would be to omit this selection condition completely from the \verb!WHERE! clause resulting in a possibly non-empty answer that would 
contain names of companies from all industrial sectors that suffice the stated average salary condition. A more relevant answer for the user could be
obtained by first generalizing the sub-sector farming to the broader sector of agriculture that includes for example fishing companies as well as the
forestry sector. Thereby, not all industrial sectors, which include companies that might have nothing in common with the intended agricultural companies,
are considered but only those that are of the user's interest. The higher relevance of the informational content from an answer of the database to an in 
this way rewritten query compared to the first approach is achieved by allowing \emph{similar} answers in case of query failure.
\end{exmp}

As shown previously in Example~\ref{sec:intro_exmp}, the \emph{similarity-based, flexible query answering} \citep{Wiese2014} enables the database in case 
of a failing query, which is possibly not satisfying for the user, to search for answers in closely related sectors regarding the chosen attribute, 
the \emph{relaxation attribute} \citep{Wiese2014}, by taking those values of the domain of the relaxation attribute into account that actually have some 
relatedness to the initally stated query which suffices to satisfy the user's needs. In contrast to this, an extension of the query sent to the
database without any modified selection condition on the relaxation attribute might have answers that have not necessarily any relatedness to the initial
query, thus, they are not really informative for the user unless he filters out those tuples matching his question most.


\defcitealias{UMLS::Sim}{UMLS::Similarity}
The "HIGHmed" project \citep{Haarbrandt2018} is an approach to develop an open infrastructure for information exchange and data management and integration
in order to facilitate and improve research and health care as a cooperation of three German university hospitals and the German Cancer Research Center
(DKFZ) as the core partners together with other partners, e.g. from research institutions or from the industry (cf. \cite[Table~2]{Haarbrandt2018}.
Indeed, the similarity plays a big role today in the health care section as indicated in the oncology use case in the article to the project.
For this, the developed platform will incorporate a "virtual oncology center" (VOC) \citep{Haarbrandt2018} with the goal to improve clinical research and 
decision making regarding cancer diseases by a functional infrastructure. This part of the platform aims to utilize shared data about courses of
treatment of patients and historical data from studies, which are provided from different participating sources, with applications built on this platform.
By doing so, the applications support the identification of similar cases for selected patients in order to help to decide for an appropriatly tailored
treatment and to be able to draw conclusions regarding the disease progression. The used similarity measures enabling the search for similar patients,
whose pathologies and treatments lead to inference of useful knowledge for the decision making, depend on machine learning and ontologies. The latter ones
also form the core part for the tool of the \citetalias{UMLS::Sim} module, developed by \cite{McInnes2009}, which will be used later in the underlying
implementation of the distributed database system to calculate similarities between disease terms according to the Medical Subject Headings taxonomy that 
are required for the clustering-based fragmentation as well as the similarity-based query answering.



\todo[inline]{Summarize content of the following sections \dots}