%!TEX root = jsba_main.tex
% Introduction

\section{Introduction}
\label{sec:intro}

Distributed databases (DDB) gained significantly more importance over the past years as they provide physically distributed storage \cite{Oszu2002, Tan2009} 
for the ever-growing amount of data in our world. Especially, in fields where the amount of data to be hosted in a database is challenging, e.g multimedia or 
textual data occurring in a social network with millions of people or medical so-called "big data" raising from different sources to be applied in 
healthcare \cite{Lee2017}, a fragmentation and replication of this big data to multiple servers in a cloud storage infrastructure instead of one single server 
can pay off:
The data fragmentation overcomes the limitation of storage capacity of a single storage instance as the total amount of data can be split into smaller parts
and increases the performance of the query processing. Furthermore, the replication of the data fragments inside the network yields a tolerance to failures by 
preventing data loss and enabling recovery, e.g. in case of a malfunction or total failure of a single server, to guarantee the desired availability and 
reliability of the data accessed the users of the distributed database system (DDBS).

Clustering-based fragmentation

Flexible Query Answering

...?