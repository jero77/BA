%!TEX root = jsba_main.tex
% Introduction

\section{Introduction}
\label{sec:intro}

\emph{Distributed databases} (DDB) gained significantly more importance over the past years as they provide physically distributed storage \citep{Tan2009} 
for the ever-growing amount of data in our world. Especially, in fields where the amount of data to be hosted in a database is challenging, e.g. multimedia or 
textual data occurring in a social network with millions of people or medical so-called "big data" retrieved and collected from different sources to be used in
healthcare for e.g. predictions or research as mentioned in \citet{Lee2017}, a fragmentation and replication of this big data to multiple servers in a cloud 
storage infrastructure instead of one single server can pay off. The data fragmentation overcomes the limitation of storage capacity of a single storage
instance as the total amount of data can be split into smaller parts, so-called fragments, and fragmentation can increase the performance of the query 
processing. 
Furthermore, the replication of the data fragments inside the network yields a tolerance to failures by compensating data loss with recovery, e.g. in case of 
a malfunction or total failure of a single server, to guarantee the desired availability and reliability of the data accessed by the users of the distributed 
database system (DDBS).


In the case of a relational data model underlying the database, a fragmentation of the data can be a \emph{horizontal} or \emph{vertical fragmentation} of the
relations (tables) of the database \citep[p.~104f.]{Ozsu1991}. A horizontal fragmentation means a tuple-wise assignment to fragments, i.e. fragments contain 
subsets of rows of the non-fragmented table, whereas a vertical fragmentation "slices" the relation by its attributes, i.e. fragments contain projections 
to the columns. 
The here analyzed and implemented approach uses a clustering heuristic on one chosen attribute as described in \citet{Wiese2014} to obtain a primary
horizontal fragmentation for a relation. The clustering procedure depends on a similarity calculation of the values of the chosen relaxation attribute 
resulting in a clustering of the column's values, such that values with a high similarity belong to the same cluster. Hence, a fragmentation of relations that 
have common attributes with this horizontally fragmented relation, i.e. attributes that will be used in queries to join these two relations, can be 
\emph{derived}~\citep[p.~116f.]{Ozsu1991}, such that the tuples which are about to be joined reside on the same server allowing for better performance in 
queries compared to query processing with a distributed join which has to be applied if the data is not residing at the same site~\citep{Sattler2009}.
Thus, this derived horizontal fragmentation improves data locality~\citep{Wiese2014} because the probably high delay of data transmission over the network, 
which is needed in case of a distributed join, will be avoided to the joy of any database user.


\defcitealias{MeSH}{MeSH\textcopyright}
The example of a \emph{hospital information system}, which is used as an application of such a distributed database system with a clustering-based 
fragmentation, is a modified version of the example defined in \citet{Wiese2014}. Patient's personal information as well as the diseases they suffer from are 
contained in the distributed database. The clustering procedure is applied to disease information of patients, whereas the disease terms are conform to the 
vocabulary provided by the \emph{Medical Subject Headings} taxonomy (see \citetalias{MeSH}) of the U.S. National Library of Medicine. 
The \emph{Unified Medical Language System} (UMLS), which was also developed and maintained by the U.S. National Library of Medicine, brings together several 
taxonomies, like the MeSH taxonomy, in order to integrate and link their biomedical and health-related vocabularies allowing applications to interchange
information based on different vocabularies.
A similarity measure defined between these disease terms yields the needed similarity calculation for the clustering-based fragmentation. In this way,
tuples holding information about patients suffering from a similar illness are stored on the same site and can be queried together in an intelligent, 
non-distributed fashion.


Upon query failure, i.e. the database cannot give an exact answer to a query formulated by a user, an empty result is returned. Assuming the query was 
formulated correctly, the blank result indicates to the user that the information he/she was looking for is not present in the current database state due to
the lack of any exact answers, and so this answer is non-informative for the user. \emph{Flexible query answering}, on the contrary, can overcome this lack of
information when the database system uses techniques like \emph{query generalization} to provide the user a non-empty result, which on the one hand does not 
match the query exactly but on the other hand contains information instead that may be relevant to the user. 

\todo[inline]{Flexible query answering (later)}

\todo[inline]{\ldots?}