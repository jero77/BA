%!TEX root = jsba_main.tex
% Discussion and Conclusion

\section{Discussion and Conclusion}
\label{sec:disc}

The previously depicted results of the query executions in the different implementations show that the clustering-based fragmentation of the data improves the
execution time of queries against the DDB significantly when comparing to the reference implementation that provides only an arbitrary, but balanced horizontal
fragmentation of the data. The intelligent similarity-based answering enables the DDBS to speed up the execution of queries containing selection conditions on
the relaxation attribute as the underlying clustering is not only meaningful but also useful when answering queries. In addition to this, the developed approaches
and the similarity-based answering techniques scale well with the size of the database if the application of similarity-based answering facilitates the execution 
enormously whereas the reference implementation's query execution times do not scale well with the size. The impact of this would be even better noticeable for
much bigger data sets revealing the strengths of the developed approaches in comparison to the reference implementation. The combination of similarity-based and
flexible query answering also outperforms the reference implementation as the clustering-based fragmentation can be brought into the query execution in an 
intelligent and efficient fashion allowing for faster execution of generalized queries compared to the rewritten and generailzed query for an implementation
without the clustering. On the other hand, there are also queries that can not be answered with the help of the proposed methods and, thus, can not make additional
improvements to the execution of this query. In some cases queries even require for some extra rewriting to fit to the adapted database schema (materialized
fragment approach) resulting in complex and long queries that do not suit the technical characteristics of the underlying database engine which produces longer
execution times.

% Security
As the implemented medical information system is only exemplary and the stored medical data is randomly generated without any connection to real data, security and
privacy do not matter here. In contrast to this, security and privacy play a big role in real medical or healthcare applications that deal with sensitive patient
data. Especially, if data is stored in a distributed infrastructure that is connected via a network and can be accessed remotely, the system offers many targets
to attackers and security mechanisms are required to guarantee privacy regarding the sensitive information. One possibility to achieve more secure communication 
for the information system is provided by \citetalias{Ignite} itself which, if enabled, allows that the Ignite nodes are connected securely by a configurable 
\emph{SSL socket communication}. Additionally, \emph{authentication} mechanisms can be activated to secure the Ignite cluster such that users have to authorize 
themselves with credentials if they want to access the cluster and the stored data. Currently, this feature is only available for a configuration of the cluster
supporting the native persistent storage. Same goes for the \emph{Transparent Data Encryption} (TDE) -- another provided security mechanism -- that allows to 
encrypt all data that is stored in a table or any backup file.

% criticism umls sim
The implemented information system demonstrates the capabilities of similarity-based and flexible query answering for medical data under a given similarity metric
for pairwise similarities of diseases. In \cite{Mathur2012} similarity metrics that base on a hierarchical relationship between entities are criticized as there
are multiple ways to build hierarchies between the entities based on different criteria such that the resulting pairwise similarity depends on the underlying
ontology and thus can yield several similarity values for a pair of entities. Furthermore, \cite{Mathur2012} state that common-ancestor-based similarities, e.g.
the Wu \& Palmer measure \citep{Wu1994}, can not correctly estimate the similarity value of two related entities that are far apart of one another in the topology.
As this is only a sample medical informational system, the used \citetalias{UMLS::Sim} measure \citep{McInnes2009} is sufficient to demonstrate the power and 
efficiency of similarity-based query answering in combination with a clustering-based fragmentation. For more realistic applications, better suited disease 
similarity metrics could be used to provide the best possible estimations for pairwise similarities. Additionally to this, more complex similarity measures for
flexible answering of queries, e.g. in order to identify not only patients that suffer from the same or similar diseases but also whole patient profiles based on
the similarity of their personal characteristics, e.g. their age, height, weight, nationality, gender, etc., and some other recorded measurements, e.g. body
temperature, blood parameters, EEG data, etc., to the examined patient, could be used for gaining more and more relevant information. When still using the proposed
techniques with a clustering-based fragmentation, this information could be queried efficiently as the search could be restricted, too, to only comprise those
fragments or partitions that contain relevant data. By doing so, the identified similar patients would still be semantically close to the examined one and not
overgeneralized, i.e. searching for similar patient profiles under all patients that have a similar or even the same disease makes sense, but searching under all
patients that suffer from some not closely related disease compared to the examined patient's disease is pointless.


% technical improvements: on-the-fly similarities, full SQL, replication
From the technical side, further improvements to the medical information system could be made with a on-the-fly computation of the pairwise similarities of the 
MeSH disease terms instead of a precalculation of all similarity values via the \citetalias{UMLS::Sim} web interface 
(\url{http://maraca.d.umn.edu/cgi-bin/umls_similarity/umls_similarity.cgi}). In order to achieve this, the disease taxonomy could be stored in a relation in the
database as well allowing to program a custom SQL function which is also supported by Ignite's SQL engine. The function could read from the taxonomy data and 
compute the similarity between to given disease terms, e.g. with an implementation of the shortest path length metric (cf. Section~\ref{sec:impl_clust}). This 
would also cancel the restriction of the term set size such that all disease terms of the MeSH data can be used in the database instead of only some of them.
Furthermore, the system only supports conjunctive queries without any subqueries or other complex constructs. This restriction could also be relaxed in order to
achieve full SQL-compatibility and a more powerful information system. Another desirable feature for the medical information system is replication of the stored 
data, which is also issued in \citet{Wiese2014}, that guarantees fault-tolerance to a certain degree and improve reliability or accessibility.