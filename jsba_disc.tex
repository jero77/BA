%!TEX root = jsba_main.tex
% Discussion

\section{Discussion}
\label{sec:disc}

\begin{enumerate}
    \item opportunity to show that you have understood the significance of your findings and that you are capable of applying theory in an independent manner
    \item investigate a phenomenon from several different perspectives
    \item question your findings
    \item consider different interpretations
\end{enumerate}


% Security
As the implemented medical information system is only exemplary and the stored medical data is randomly generated without any connection to real data, security and
privacy do not matter here. In contrast to this, security and privacy play a big role in real medical or healthcare applications that deal with sensitive patient
data. Especially, if data is stored in a distributed infrastructure that is connected via a network and can be accessed remotely, the system offers many targets
to attackers and security mechanisms are required to guarantee privacy regarding the sensitive information.

% criticism umls sim
The implemented information system demonstrates the capabilities of similarity-based and flexible query answering for medical data under a given similarity metric
for pairwise similarities of diseases. In \cite{Mathur2012} similarity metrics that base on a hierarchical relationship between entities are criticized as there
are multiple ways to build hierarchies between the entities based on different criteria such that the resulting pairwise similarity depends on the underlying
ontology and thus can yield several similarity values for a pair of entities. Furthermore, \cite{Mathur2012} state that common-ancestor-based similarities, e.g.
the Wu \& Palmer measure \citep{Wu1994}, can not correctly estimate the similarity value of two related entities that are far apart of one another in the topology.
As this is only a sample medical informational system, the used \citetalias{UMLS::Sim} measure \citep{McInnes2009} is sufficient to demonstrate the power and 
efficiency of similarity-based query answering in combination with a clustering-based fragmentation. For more realistic applications, better suited disease 
similarity metrics could be used to provide the best possible estimations for pairwise similarities. Additionally to this, more complex similarity measures for
flexible answering of queries, e.g. in order to identify not only patients that suffer from the same or similar diseases but also whole patient profiles based on
the similarity of their personal characteristics, e.g. their age, height, weight, nationality, gender, etc., and some other recorded measurements, e.g. body
temperature, blood parameters, EEG data, etc., to the examined patient, could be used for gaining more and more relevant information. When still using the proposed
techniques with a clustering-based fragmentation, this information could be queried efficiently as the search could be restricted, too, to only comprise those
fragments or partitions that contain relevant data. By doing so, the identified similar patients would still be semantically close to the examined one and not
overgeneralized, i.e. searching for similar patient profiles under all patients that have a similar or even the same disease makes sense, but searching under all
patients that suffer from some not closely related disease compared to the examined patient's disease is pointless.