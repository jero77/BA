% Expose zur Bachelorarbeit

\documentclass[a4paper, 11pt]{article}

% Packages
\usepackage[utf8]{inputenc}
\usepackage[T1]{fontenc}
\usepackage{lmodern}
\usepackage[german]{babel}
\usepackage{moreverb}
\usepackage{textcomp}
\usepackage{url}
    % natbib
\usepackage[square,numbers]{natbib}
\usepackage[nottoc]{tocbibind}
\bibliographystyle{unsrtnat}
\setcitestyle{authoryear}
    % math
\usepackage{mathtools}

\usepackage{caption}



\selectlanguage{german}
\begin{document}
%Titelseite
\newpage
\thispagestyle{empty}
\newcommand{\Rule}{\rule{\textwidth}{1mm}}

\begin{center}

\Rule\vspace{5mm}
\sffamily\bfseries\LARGE Exposé \par
\sffamily\bfseries\Huge Similarity-based query answering for medical information systems
\vspace{1mm}\Rule
\vfill
\sffamily\bfseries\LARGE Bachelor-Thesis
\vfill
\sffamily\bfseries\Large Jero Mario Schäfer\par Matrikelnr.: 21552103\par Angewandte Informatik\par
\vfill

\raisebox{7mm}{Georg-August-Universität}
%\includegraphics[height=xxmm]{unilogo.pdf}
\raisebox{7mm}{Göttingen}\par
\vfill
Abgabedatum: 01.04.19
\end{center}




\newpage
\section{Einleitung}

Medizinische Informationssysteme dienen der Speicherung, Auswertung und Verarbeitung patientenbezogener Daten. So könnte zum Beispiel eine einfache Datenbank
wie folgt aussehen. In der Tabelle~\ref{tab:info} sind die persönlichen Informationen der Patienten wie z.B. Name und Alter gespeichert. Die 
Patienten besitzen zudem einen eindeutigen Identifikator, der zufällig bestimmt werden kann. Des Weiteren gibt es zusätzlich zu den persönlichen Daten noch
die Tabelle~\ref{tab:disease}, die sowohl Auskunft über die Erkrankungen geben, an denen die zu behandelnden Patienten leiden, als auch bspw. Daten über die
durchgeführten bzw. durchzuführenden Behandlungen enthalten könnte: 
\\
\newcommand{\textoverline}[1]{$\overline{\mbox{#1}}$}
\begin{center}
\begin{tabular}{*{4}l} 
     \underline{ID} & Name & Address & Age \\
     \hline
     1951 & John Miller & New street 5, 44579 Newtown & 28 \\
     4431 & Mary Smith & Green way 15, 23240 Abcity & 47 \\
     0056 & Taylor Banks & On the line 90, 11005 Nington & 71 \\
\end{tabular}
\captionof{table}{Patienten Informationen} \label{tab:info}
\end{center}

\begin{center}
\begin{tabular}{*{4}l} 
     \underline{ID} & \textoverline{PatientID} & Disease & Treatment\\
     \hline
     92840 & 1951 & Cough & Antitussive \\
     12748 & 4431 & Influenza & Antiviral Medication\\
     12091 & 0056 & Bronchitis & Bronchodilator \\
     75634 & 1951 & Asthma &  Inhalator \\
\end{tabular}
\captionof{table}{Erkrankungen} \label{tab:disease}
\end{center}



Dabei referenziert ein Tupel der Tabelle über Erkrankungen jeweils einen Patienten mittels seines Identifikators und stellt somit den Zusammenhang zwischen 
den beiden Relationen her. Die Primärschlüssel sind in den Tabellenabbildungen durch Unterstreichungen und die Fremdschlüssel durch Überstreichungen 
gekennzeichnet.


Ein solches Informationssystem ist zum Einen nützlich und sogar notwendig für die behandelnden Ärzte und Krankenpfleger innerhalb eines Krankenhauses, aber zum
Anderen auch wichtig für die administrativen Aufgaben, die in einem Krankenhaus anfallen. Außerdem könnten auch Hausärzte oder Fachärzte außerhalb eines 
Krankenhauses durch Zugriff auf die Daten über die Erkrankungen ihrer Patienten profitieren. Da die anfallenden Datenmengen mitunter sehr groß sein können, 
bietet es sich an nicht nur eine einzelne Datenbank zu haben sondern ein verteiltes Datenbanksystem zu nutzen, das die großen Datenmengen durch die 
physikalische Aufteilung der Daten auf verschiedene Server, welche alle eine Datenbank bzw. einen Teil der gesamten, verteilten Datenbank beherbergen und
verwalten, handhaben kann. Um die Daten hierbei sinnvoll aufzuteilen, kann man bspw. eine horizontale Fragmentierung, also eine zeilenweise Aufteilung der 
Tabellen auf verschiedene Server, anwenden. Die dadurch resultierende Partitionierung der Relationen stellt eine Unterteilung der Daten in kleinere Teilmengen
dar, die letztendlich über das Netzwerk verteilt werden können.


Ziel dieser Arbeit ist es, anhand des Beispiels eines medizinischen Informationssystems ein verteiltes Datenbanksystem zu implementieren, welches aufgrund 
eines Maßes für Ähnlichkeit von Krankheiten ein \emph{Clustering} der Erkrankungsrelation bestimmt, um darauf aufbauend eine horizontale Fragmentierung der 
Daten zu erhalten und infolgedessen eine abgeleitete Fragmentierung der persönlichen Daten der Patienten zu ermitteln, die bewirkt, dass die Tupel, welche 
Namen, Adresse, Alter, etc. enthalten, zusammen mit den Informationen über die Erkrankungen eines Patienten abgespeichert werden.
Dies ermöglicht eine intelligente Anfragebearbeitung, bei der Informationen aus den beiden Relationen mittels eines Joins lokal verknüpft werden können, 
da die Tupel nicht erst über das Netzwerk von einem zum nächsten Datenbankserver gesendet werden müssen, wie es der Fall sein könnte, wenn die Daten nicht
zusammen abgelegt werden. Des Weiteren soll dieses System im Falle einer Anfrage mit einer Selektionsbedingung auf dem Krankheitsattribut der 
Tabelle~\ref{tab:disease}, sofern sie kein exaktes Ergebnis liefern kann, dem Nutzer ermöglichen, die Anfrage \emph{flexibel} mit Hilfe der Ähnlichkeit 
zwischen Erkrankungen zu beantworten. Dabei wird die Anfrage des Nutzers generalisiert, wodurch ggf. Antworten von der Datenbank zurückgeliefert werden können,
sodass der Nutzer zu der ursprünglich gestellten Anfrage ähnliche Antworten erhalten kann, die aus seiner Sicht nützliche und wertvolle Informationen 
darstellen könnten.


\defcitealias{Ignite}{Apache Ignite\texttrademark}
\defcitealias{MeSH}{Medical Subject Headings (MeSH\textcopyright)}
\defcitealias{UMLS::Sim}{UMLS::Similarity}
\defcitealias{UMLS}{Unified Medical Language System (UMLS)}
Die Implementation soll auf Basis einer \citetalias{Ignite} Datenbank (siehe~\url{https://ignite.apache.org/}) erfolgen. Die beispielhaften Patientendaten 
werden randomisiert erzeugt und gespeichert, wobei die Krankheiten auf dem in der \citetalias{MeSH}
(siehe~\url{https://meshb.nlm.nih.gov/search}) Datenbank der U.S. National Library of Medicine festgelegten Vokabular basiert. Zur Bestimmung der Ähnlichkeit
zweier Terme, die Erkrankungen beschreiben, wird ein Tool des \citetalias{UMLS::Sim} Moduls (siehe~\url{http://umls-similarity.sourceforge.net/}) herangezogen.
Das Modul unterstützt dabei die Ontologien, wie bspw. MeSH, des \citetalias{UMLS}.
Die Arbeit basiert maßgeblich auf dem Artikel \citet{Wiese2014}, in dem die theoretischen Grundlagen sowie praktische Aspekte zu dem Thema untersucht werden.


\newpage
\bibliography{literature}
\end{document}}